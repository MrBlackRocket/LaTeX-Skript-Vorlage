% Schriftauswahl als Standard ist Palantino/Helvetica/Courier
% gewählt alternativ kann aber Times/Helvetica/Courier oder
% Droid Schrift Familie verwendet werden, dafür muss 
% 'palantino' durch 'times' bzw. 'droid' ersetzt werden. Bei 
% Auswahl der Droid Schrift sollte diese auch installiert sein
% (http://www.ctan.org/tex-archive/fonts/droid/)
\Schrift{palantino}

% Sollen alle Abürzungen angezeigt werden?
% Standart ist 'nein'
\AlleAbkuerzungenAuflisten{nein}

% Titel des Scripts
\title{Öffentlichen Recht III}

% Einstellung des Logos
% Wenn kein Logo verwendet werden soll, dann muss
% './graphics/unilogo.png' durch './graphics/das-nichts.png'
% ersetzt werden.
\makeatletter
	\newcommand*\@logo{./grafiken/unilogo.png}
\makeatother

% Ersteller/Autor des Scripts
\author{Max Musterman}

% Erstellungsort
\makeatletter
	\newcommand*\@ort{Musterstadt}
\makeatother

% eine kurze Beschreibung des Scriptes
\subject{Mitschrift aus Vorlesung XY}

% Schlüsselwörter
\makeatletter
	\newcommand*\@kwords{LaTeX, Vorlage, Script, CC-BY-SA 3.0}
\makeatother

% Anschrift
\makeatletter
	\newcommand*\@kontakt{
		Max Musterman\\ 
		Musterstraße 42\\ 
		31415 Musterstadt\\ \href{mailto:my_address@Domain.de}{\nolinkurl{my_address@Domain.de}}\\ \href{http://www.Domain.de}{www.Domain.de}
	}
\makeatother

% Lizenz
% Als Standard wird die CC-BY-SA 3.0 Lizenz angezeigt, bei einer
% anderen Lizenzwahl das Logo und des Link ersetzen. Soll keine
% Lizenz angezeigt werden, so muss die Zeile mit des '\href' mit
% einem '%'-Zeichen auskommentiert werden. 
\makeatletter
	\newcommand*\@lizenz{
		\href{http://creativecommons.org/licenses/by-sa/3.0/deed.de}{\includegraphics[width=1in]{./grafiken/by-sa.png}}
	}
\makeatother