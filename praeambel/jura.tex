% ========
% Jura

%**************************************************************
% Entnommen jur-rdnr.sty
% Peter Felix Schuster (http://www.peterfelixschuster.de)
%**************************************************************
\makeatletter
% Zähler für Fälle und Abwandlungen definieren und setzen
\newcommand{\@rdnrname}{Rn.\,} % Makro für die Bezeichnung für Fälle, macht es flexibel zu ändern
\newcounter{randnr} % Zähler randnr (für Randnummer)
\newcounter{zwrandnr}[randnr] % Zähler zwrandnr (für Zwischen-Randnummer), eine Erhöhung von randnr setzt ihn zurück
\setcounter{randnr}{0} % Anfangs auf 0 setzen
\setcounter{zwrandnr}{0} % Anfangs auf 0 setzen
\newcommand{\newrandnummer}{\refstepcounter{randnr}} % Makro, das den Zähler erhöht
\newcommand{\newzwischenrandnummer}{\refstepcounter{zwrandnr}} % Makro, das den Zähler erhöht
\renewcommand{\therandnr}{\arabic{randnr} }% In arabischen Ziffern nummerieren.
\renewcommand{\thezwrandnr}{\therandnr\,\alph{zwrandnr}} % Kleinbuchstaben nach ''normaler'' Randnummer.
\renewcommand{\p@randnr}{\@rdnrname} % Bei Verweisen Rdnr. vor Zahl ausgegeben
\renewcommand{\p@zwrandnr}{\@rdnrname} % Bei Verweisen Rdnr. vor Zahl\,Buchstabe ausgegeben

% marginpar Config
\setlength{\marginparwidth}{15mm}
\marginparsep5mm
\normalmarginpar

% Zähler über Makro erhöhen und am Rand als Randnummer ausgeben
\newcommand{\randnummer}{\newrandnummer\marginpar{\vspace{-4pt}{\strut\\\centering\textbf{\therandnr}}}}

% Makro für Zwischenrandnummern
\newcommand{\zwischenrandnummer}{\newzwischenrandnummer\marginpar{\strut\\\centering\textbf{\thezwrandnr}}}
\makeatother
%**************************************************************

% Formatierung für nicht nummerierte Listen 
\makeatletter
\renewcommand{\labelitemi}{$\m@th\bullet$} 
\renewcommand{\labelitemii}{--}
\renewcommand{\labelitemiii}{$\m@th\circ$}
\renewcommand{\labelitemiv}{$\m@th\rightarrow$}
\makeatother

% Formatierung für nummerierte Listen [I, 2, c) dd)]
\renewcommand{\labelenumi}{\Roman{enumi}.}
\renewcommand{\labelenumii}{\arabic{enumii}.}
\renewcommand{\labelenumiii}{\alph{enumiii})}
\renewcommand{\labelenumiv}{\alph{enumiv}\alph{enumiv})}



%Abkürzungen - vornehmlich für schmale Zwischenräume (\,) nützlich
%benötigt für einige Befehle jurabib. Befehl nachschlagen, mit dem überprüft werden kann, ob geladen!
\providecommand{\Abs}[1]{Abs.\,#1}
\providecommand{\aF}{a.\,F.\xspace} %\aF => a. F.
\providecommand{\andM}{\textbf{a.\,M.}\xspace} %\andM => a. M. in Fettdruck
\providecommand{\euro}[1]{#1\,\EUR} %\euro{x} => x (Eurosymbol)
\providecommand{\folg}{\,f.\xspace} %f. nach kurzem Leerraum
\providecommand{\ffolg}{\,ff.\xspace} %ff. nach kurzem Leerraum
\providecommand{\hL}{h.\,L.\xspace} %\hL => h.L.
\providecommand{\hM}{h.\,M.\xspace} %\hM => h. M.
\providecommand{\idR}{i.\,d.\,R.\xspace} %\idR => i. d. R.
\providecommand{\iHv}{i.\,H.\,v.\xspace} %\iHv => i. H. v.
\providecommand{\iSd}{i.\,S.\,d.\xspace} %\iSd => i. S. d.
\providecommand{\iSv}{i.\,S.\,v.\xspace} %\iSv => i. S. v.
\providecommand{\iVm}{i.\,V.\,m.\xspace} %\iVm => i. V. m.
\providecommand{\ja}{\ding{'63}\xspace}%\ensuremath{\bigoplus}%(+) 
\providecommand{\mwN}{m.\,w.\,N.\xspace} %\mwN => m. w. N.
\providecommand{\LWL}{\bfseries\underline{\emph{LWL}}}%steht für Literaturwunschliste, also für Blindzitate. Alles folgende wird fett!
\providecommand{\nein}{\ding{'67}\xspace}%\ensuremath{\ominus}%(-)
\providecommand{\nF}{n.\,F.\xspace} %\nF => n. F.
\providecommand{\Nr}[1]{Nr.\,#1}
\providecommand{\Nro}[1]{Nr.\,#1}
\providecommand{\pg}[1]{\S\,#1} %\pg{x} => (Paragraf) x
\providecommand{\Pg}[1]{\SSS\,#1} %\Pg{x} => (Paragrafen) x 
\providecommand{\pgAbs}[2]{\S\,#1 Abs.\,#2} %\pgAbs{x}{y} => (Paragraf) x Abs. y
\providecommand{\pgAbsS}[3]{\S\,#1 Abs.\,#2 S.\,#3}
\providecommand{\pgRn}[2]{\S\,#1 Rn.\,#2} %\pg{x}{y} => (Paragraf) x Rn. y
\providecommand{\pgS}[2]{\S\,#1 S.\,#2}
\providecommand{\Rn}[1]{Rn.\,#1}
\providecommand{\Satz}[1]{S.\,#1}
\providecommand{\Seite}[1]{S.\,#1}
\providecommand{\zBsp}{z.\,B.}
\providecommand{\fremdwort}[2]{\foreignlanguage{#1}{\itshape{}#2}}%\fremdwort{#1=Sprache}{#2=Text}
\newcommand*{\qll}[1]{\emph{#1}}%           fuer Quellen
\newcommand{\code}[1]{\texttt{#1}}%         fuer Computeranweisungen, tags o.ae.
\newcommand*{\marke}[1]{{\scshape #1}}%     Markennamen % \texttrademark (TM) oder \textregistered (R)?
\newcommand*{\firma}[1]{{\scshape #1}}%     Unternehmensbezeichnung
\newcommand*{\prdbez}[1]{%                  Produktbezeichnung
  {\scshape #1}%
  \index{#1}%
}
\providecommand{\altzahl}[1]{{\oldstyle #1}}
%\DeclareTextSymbol{\textNr}{TS1}{'233}

\newcommand*{\arr}{\(\rightarrow\)\space}
\newcommand*{\arrr}{\(\longrightarrow\)\space}
\newcommand*{\Arr}{\(\Rightarrow\)\space}
\newcommand*{\Arrr}{\(\Longrightarrow\)\space}
\newcommand*{\lrarr}{\(\leftrightarrow\)\space}

\newcommand*{\larr}{\(\leftarrow\)}
\newcommand*{\Larr}{\(\Leftarrow\)}

\newcommand*{\darr}{\(\downarrow\)}
\newcommand*{\Darr}{\(\Downarrow\)}
