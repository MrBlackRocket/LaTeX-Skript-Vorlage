%----------------------------------
% laden der Verschiedenen Pakete
%----------------------------------
\input{./praeambel/preambelcommands.tex} % Quelle http://brcs.eu/d0fio

% Schrift & Sprache
\usepackage[utf8]{inputenc} % Schriftsatz utf8 verwenden 
\usepackage[T1]{fontenc} % T1-Fonts
\usepackage[ngerman]{babel} % Deutsche Sonderzeichen und Silbentrennung (neue Rechtschreibung)
\usepackage{marvosym} % diverse Symbole
\usepackage{wasysym} % diverse Symbole 
\usepackage{eulervm} % Mathematische Symbole 
\usepackage{amssymb} % Mathematische Symbole 
\RequirePackage{pifont} % einige Befehle brauchen pifont (Symbole)
\usepackage[babel,german=guillemets]{csquotes} % franz. Anführungszeichen verwenden
\usepackage{rotating} % erlaubt es Text zu drehen

% Layout Pakete
\usepackage{xspace} % behebt Makrogrenz-Problem
\usepackage{geometry} % für Seitenränder, 
\usepackage{setspace} % für Zeilenabstand
\usepackage{calc} % Berechnen von Längen und Werten
\usepackage{microtype} % Automatische Schrift vs. Layout Anpassungen
\usepackage{scrpage2} % für Kopf- und Fußzeilen

% Tabellen
\usepackage{tabularx} % bessere Tabellen
\usepackage[para]{threeparttable} %Fussnoten innerhalb von Tabellen
\usepackage{array} % wird von colortbl benötigt
\usepackage{colortbl} % erlaubt Farbige Zellen
\usepackage[table]{xcolor} % alternierende farbige Tabellenzeilen
\usepackage{booktabs} % Gestaltung der horizontalen Linien innerhalb einer Tabelle
\usepackage{multirow}
\usepackage{longtable}

% JURA spezifische Pakete
\usepackage{jurabib} % juristisches Literaturverzeichnis, Optionen mit \jurabibsetup
\newcommand{\AlleAbkuerzungenAuflisten}[1]{\ifthenelse{\equal{#1}{ja}}{\usepackage{acronym}}{}\ifthenelse{\equal{#1}{nein}}{\usepackage[printonlyused]{acronym}}{}} % Abkürzungsverzeichnis
\usepackage{enumitem} % Anpassbare Enumerates/Itemizes
\usepackage{marginnote} % für Randnotizen (von den Randnummern verwendet)
\usepackage{titlesec} % ermöglicht das anpassen der Titel

% andere Pakete
\usepackage{graphicx} % Grafikpaket
\usepackage{tikz} % für hervor gehobene Boxen
\usetikzlibrary{calc,arrows,shadings,shadows}
\newcommand\Loadedframemethod{tikz}
\usepackage[framemethod=\Loadedframemethod]{mdframed}


%----------------------------------
% Pakete mit Config
%----------------------------------

% Aussehen der Gliederung
\usepackage[tocgraduated]{tocstyle} % Aussehen der Gliederung
\usetocstyle{KOMAlike}

% Hyperlinks und interne PDF-Verweise
\usepackage[
       % Farben fuer die Links
       colorlinks=true,         % Links erhalten Farben statt Kaeten
       urlcolor=pdfurlcolor,    % \href{...}{...} external (URL)
       filecolor=pdffilecolor,  % \href{...} local file
       linkcolor=pdflinkcolor,  % \ref{...} and \pageref{...}
       citecolor=pdfurlcolor,
       menucolor=pdfurlcolor,
       % Links
       raiselinks=true,			% calculate real height of the link
       backref=false,           % Backlinks im Literaturverzeichnis (section, slide, page, none)
       pagebackref=false,       % Backlinks im Literaturverzeichnis mit Seitenangabe
       verbose,
       hyperindex=true,         % backlinkex index
       linktocpage=true,        % Inhaltsverzeichnis verlinkt Seiten
       hyperfootnotes=true,     % Keine Links auf Fussnoten
       % Bookmarks
       bookmarks=true,         % Erzeugung von Bookmarks
       bookmarksopenlevel=1,    % Gliederungstiefe der Bookmarks
       bookmarksopen=false,      % Expandierte Untermenues in Bookmarks
       bookmarksnumbered=true,  % Nummerierung der Bookmarks
       bookmarkstype=toc,       % Art der Verzeichnisses
       bookmarksdepth=3,
       % Anchors
       plainpages=false,        % Anchors even on plain pages ?
       pageanchor=true,         % Pages are linkable
       pdfstartview=FitH,       % Dokument wird Fit Width geaefnet
       pdfpagemode=UseOutlines, % Bookmarks im Viewer anzeigen
       pdfpagelabels=true,      % set PDF page labels
]{hyperref}

% PDF Informationen
\makeatletter
\AtBeginDocument{
  \hypersetup{
    pdftitle={\@title},
    pdfauthor={\@author},
    pdfsubject={\@subject},
    pdfkeywords={\@kwords}
  }
}
\makeatother

%----------------------------------
% Config
%----------------------------------

% Layout
% Optionen für das geometry-Paket. 
% Rand links 2,5cm, rechts 2,5cm, oben 3cm, unten 3cm.
\geometry{lmargin=25mm,rmargin=2.5cm,tmargin=30mm,bmargin=30mm,headheight=0ex}

% Hurenkinder und Schusterjungen/Waisenkinder vermeiden
\clubpenalty = 10000
\widowpenalty = 10000
\displaywidowpenalty = 10000

% löscht alle bisherigen Kopf- und Fußzeilen
\clearscrheadfoot

% selbstgebastelte Kopf-/Fußzeilen
\pagestyle{scrplain}

% Die Kopfzeile 
\ihead[]{} \chead[]{} \ohead[]{}

% Fußzeile
\ifoot{} \cfoot[\pagemark]{\pagemark} \ofoot{}


% Absätze nicht einrücken
\setlength{\parindent}{0pt}

% mdframed Global Config
\mdfsetup{skipabove=\topskip,skipbelow=\topskip}

% Schriftart
\newcommand{\Schrift}[1]
{
  \ifthenelse{\equal{#1}{droid}}{
    \usepackage[default]{droidserif}
    \usepackage[defaultsans]{droidsans}
  	\usepackage[defaultmono]{droidmono}
  	\renewcommand{\rmfamily}{\fdsfamily}
  	\addtokomafont{disposition}{\fdmfamily}
  	\setkomafont{sectioning}{\fdsfamily\bfseries\normalsize} 
  	\urlstyle{rm}
  }{}
  \ifthenelse{\equal{#1}{times}}{
  	\usepackage{mathptmx}
  	\usepackage[scaled=.95]{helvet} 
  	\usepackage{courier} 
  	\urlstyle{rm}
  }{}
  \ifthenelse{\equal{#1}{palantino}}{
    \usepackage{mathpazo}
    \usepackage[scaled=.95]{helvet}
    \usepackage{courier}
    \urlstyle{rm}
  }{}
}


% durch die boxes.tex werden verschiedene Boxen bereit gestellt
\input{./praeambel/boxen.tex}

% durch die table.tex wird eine Tabelle
% ========
% Makro für Tabellen aus dem Quellcode des freiesMagazin (http://www.freiesmagazin.de/)
\newcommand{\PDFonly}[1]{#1}
% Vor einem Abstand die Zeile strecken.
\newboolean{AbstandSoftLineBreak}

% Vor einem Abstand die Zeile nicht hart umbrechen.
\newboolean{AbstandHardLineBreak}

% Nach einem Abstand die Spalte beenden.
\newboolean{AbstandColumnBreak}

% Setzen der Abstand-Standardwerte
\newcommand*{\SetDefaultAbstandKeys}
{%
    \setboolean{AbstandSoftLineBreak}{false}%
    \setboolean{AbstandHardLineBreak}{false}%
    \setboolean{AbstandColumnBreak}{false}%
}

% Abstand im Text, der einen kompletten Bereich frei lässt.
% Benutzung: \Abstand[OPTIONEN]{HOEHE}
% Beispiel: \Abstand{0.7em} oder \Abstand[stretch]{7.1em}
% Optionen:
%   stretch - Fügt vor dem Abstand einen sanften Zeilenumbruch ein,
%             sodass eine Zeile bis zum Spaltenende gestreckt wird.
%   linebreak - Fügt einen harten Zeilenumbruch ein.
%   break - Beendet nach dem Abstand die Spalte.
\newcommand*{\Abstand}[2][]
{%
    \PDFonly{%
        \SetDefaultAbstandKeys{}%
        \setkeys{Abstand}{#1}%
        \ifthenelse{\boolean{AbstandSoftLineBreak}}{\sbreak}{}%
        \ifthenelse{\boolean{AbstandHardLineBreak}}{\\}{}%
        \vspace*{#2}%
        \ifthenelse{\boolean{AbstandColumnBreak}}{\cbr

        }{\par{}}%
    }%
}

% Abstand hinter einer Tabelle
\newlength{\TabellenVSkip}
% Setzen der Tabellen-Standardwerte
\newcommand*{\SetDefaultTabellenKeys}[1]
{%
    \setlength{\TabellenVSkip}{0em}%
}

% Makro für Dateien, Ordner, Verzeichnisse, Befehle, Optionen etc. im Terminal
\newcommand*{\term}[1]{\textbf{\texttt{#1}}}
\newcommand*{\termZ}[2]{\textbf{\texttt{#1}} \textbf{\texttt{#2}}}
\newcommand*{\termD}[3]{\textbf{\texttt{#1}} \textbf{\texttt{#2}} \textbf{\texttt{#3}}}

% Makro für Tabellen aus dem Quellcode des freiesMagazin (http://www.freiesmagazin.de/)
% Benutzung: \begin[OPTIONEN]{Tabelle}{SPALTENANZAHL}{SPALTENDEFINITION}{UEBERSCHRIFT} ... \end{Tabelle}
% Beispiel: \begin[OPTIONEN]{Tabelle}{3}{|p{1cm}cc|}{Tabelle 1} erzeugt eine dreispaltige Tabelle, wobei die erste Spalte 1 cm und linksbündig ist, die anderen beiden zentriert

\makeatletter
% Abstand hinter einer Tabelle
\define@key{Tabelle}{vskip}{\setlength{\TabellenVSkip}{#1}}
\makeatother
\newenvironment{Tabelle}[4][]
{%
    \SetDefaultTabellenKeys{}%
    \setkeys{Tabelle}{#1}%
    \begin{minipage}{\linewidth}%
        \centering{}%
        \begin{footnotesize}%
            \renewcommand{\arraystretch}{1.2}% erhöht den Zeilenabstand für Ueberschrift
            \setlength{\arrayrulewidth}{2pt}%
            \rowcolors{1}{mittelgrau}{white}% alternierende Farben
            \arrayrulecolor{darkgrey}%
            \begin{tabular}{#3}
                \firsthline
                \multicolumn{#2}{|>{\columncolor{darkgrey}}c|}%
                {\normalsize \textcolor{white}{\textbf{#4}}}\\
}
{%
                \lasthline
            \end{tabular}%
            \rowcolors{1}{white}{white}%
            \renewcommand{\arraystretch}{1}%
        \end{footnotesize}%
    \end{minipage}%
    \Abstand{\TabellenVSkip}
}

% Juristische Gliederungsebenen
%Inhaltsverzeichnis in Gliederung umbenennen
\addto\captionsngerman{%
  \renewcommand{\contentsname}{Gliederung}
}

% griechische Kleinbuchstaben für die Gliederung
\newcommand{\greek}[1]{\ensuremath{\ifcase\value{#1}\or \mbox{\boldmath$\alpha$}\or \mbox{\boldmath$\beta$}\or \mbox{\boldmath$\gamma$}\or \mbox{\boldmath$\delta$}\or \mbox{\boldmath$\varepsilon$}\or \mbox{\boldmath$\zeta$}\or \mbox{\boldmath$\eta$}\or \mbox{\boldmath$\theta$}\or \mbox{\boldmath$\iota$}\or \mbox{\boldmath$\kappa$}\or \mbox{\boldmath$\lambda$}\or \mbox{\boldmath$\mu$}\or \mbox{\boldmath$\nu$}\or \mbox{\boldmath$\xi$}\or \mbox{\boldmath$o$}\or \mbox{\boldmath$\pi$}\or \mbox{\boldmath$\varrho$}\or \mbox{\boldmath$\sigma$}\or \mbox{\boldmath$\tau$}\or \mbox{\boldmath$\upsilon$}\or \mbox{\boldmath$\varphi$}\or \mbox{\boldmath$\chi$}\or \mbox{\boldmath$\psi$}\or \mbox{\boldmath$\omega$}\fi}}

\makeatletter 
% doppelte Aufzählung
\providecommand*{\doublegreek}[1]{\greek{#1}{\greek{#1}}}
\providecommand*{\doublealph}[1]{\alph{#1}{\alph{#1}}}
\makeatother

% KOMA-Script 
\setcounter{secnumdepth}{13}   % Nummerierungstiefe Überschriften 


% \titleformat{⟨Überschriftenklasse⟩}[Absatzformatierung⟩]{⟨Textformatierung⟩} {⟨Nummerierung⟩}{⟨Abstand zwischen Nummerierung und Überschriftentext⟩}{⟨Code vor der Überschrift⟩}[⟨Code nach der Überschrift⟩]
% \titlespacing{⟨Überschriftenklasse⟩}{⟨Linker Einzug⟩}{⟨Platz oberhalb⟩}{⟨Platz unterhalb⟩}[⟨rechter Einzug⟩]

% \part
\titleformat{\part}[display]{\normalfont\bfseries\LARGE\filcenter}{\thepart}{.5em}{}
\renewcommand \thepart {}

% \chapter
\titleformat{\chapter}[hang]{\LARGE\bfseries}{\thechapter}{.5em}{}
\titlespacing{\chapter}{0pt}{-3em}{-1em}
\renewcommand \thechapter {{\S\xspace\arabic{chapter}}}

% \section
\titleformat{\section}[hang]{\Large\bfseries}{\thesection}{.5em}{}
\titlespacing{\section}{0pt}{6pt}{-1em}
\renewcommand \thesection {\Alph{section}}

% \subsection
\titleformat{\subsection}[hang]{\large\bfseries}{\thesubsection}{.5em}{}
\titlespacing{\subsection}{0pt}{6pt}{-1em}
\renewcommand \thesubsection {\Roman{subsection}.}

% \subsubsection
\titleformat{\subsubsection}[hang]{\normalsize\bfseries}{\thesubsubsection\quad}{.5em}{}
\titlespacing{\subsubsection}{0pt}{6pt}{-1em}
\renewcommand \thesubsubsection {\arabic{subsubsection}.}

% \paragraph
\titleformat{\paragraph}[hang]{\normalsize\bfseries}{\theparagraph\quad}{-5pt}{}
\titlespacing{\paragraph}{0pt}{6pt}{-1em}
\renewcommand \theparagraph {\alph{paragraph})}

% \subparagraph
\titleformat{\subparagraph}[hang]{\normalsize\bfseries}{\thesubparagraph\quad}{-5pt}{}
\titlespacing{\subparagraph}{0pt}{6pt}{-1em}
\renewcommand \thesubparagraph {\doublealph{paragraph})}

% \subsubparagraph
\titleclass{\subsubparagraph}{straight}[\subparagraph]
\titleformat{\subsubparagraph}[hang]{\normalsize\bfseries}{\greek{subsubparagraph})}{.5em}{}[]
\titlespacing{\subsubparagraph}{0pt}{6pt}{-1em}
\newcounter{subsubparagraph}[subparagraph]
\renewcommand \thesubsubparagraph {\greek{subsubparagraph})}

% \fourfoldparagraph
\titleclass{\fourfoldparagraph}{straight}[\subsubparagraph]
\titleformat{\fourfoldparagraph}[hang]{\normalsize\bfseries}{\doublegreek{fourfoldparagraph})}{.5em}{}[]
\titlespacing{\fourfoldparagraph}{0pt}{6pt}{-1em}
\newcounter{fourfoldparagraph}[subsubparagraph]
\renewcommand \thefourfoldparagraph {\doublegreek{fourfoldparagraph})}

% \fourfoldparagraph
\titleclass{\fivefoldparagraph}{straight}[\fourfoldparagraph]
\titleformat{\fivefoldparagraph}[hang]{\normalsize\bfseries}{(\arabic{fivefoldparagraph})}{.5em}{}[]
\titlespacing{\fivefoldparagraph}{0pt}{6pt}{-1em}
\newcounter{fivefoldparagraph}[fourfoldparagraph]
\renewcommand \thefivefoldparagraph {(\arabic{fivefoldparagraph})}

% \sixfoldparagraph
\titleclass{\sixfoldparagraph}{straight}[\fivefoldparagraph]
\titleformat{\sixfoldparagraph}[hang]{\normalsize\bfseries}{(\alph{sixfoldparagraph})}{.5em}{}[]
\titlespacing{\sixfoldparagraph}{0pt}{6pt}{-1em}
\newcounter{sixfoldparagraph}[fivefoldparagraph]
\renewcommand \thesixfoldparagraph {(\alph{sixfoldparagraph})}

% \sevenfoldparagraph
\titleclass{\sevenfoldparagraph}{straight}[\sixfoldparagraph]
\titleformat{\sevenfoldparagraph}[hang]{\normalsize\bfseries}{(\doublealph{sevenfoldparagraph})}{.5em}{}[]
\titlespacing{\sevenfoldparagraph}{0pt}{6pt}{-1em}
\newcounter{sevenfoldparagraph}[sixfoldparagraph]
\renewcommand \thesevenfoldparagraph {(\doublealph{sevenfoldparagraph})}

% \eightfoldparagraph
\titleclass{\eightfoldparagraph}{straight}[\sevenfoldparagraph]
\titleformat{\eightfoldparagraph}[hang]{\normalsize\bfseries}{(\greek{eightfoldparagraph})}{.5em}{}[]
\titlespacing{\eightfoldparagraph}{0pt}{6pt}{-1em}
\newcounter{eightfoldparagraph}[sevenfoldparagraph]
\renewcommand \theeightfoldparagraph {(\greek{eightfoldparagraph})}

% \eightfoldparagraph
\titleclass{\ninefoldparagraph}{straight}[\sevenfoldparagraph]
\titleformat{\ninefoldparagraph}[hang]{\normalsize\bfseries}{(\doublegreek{ninefoldparagraph})}{.5em}{}[]
\titlespacing{\ninefoldparagraph}{0pt}{6pt}{-1em}
\newcounter{ninefoldparagraph}[sevenfoldparagraph]
\renewcommand \theninefoldparagraph {(\greek{ninefoldparagraph})}

% richtig formatiet in der Gleiderung Ausgeben
\makeatletter
\newcommand{\l@subsubparagraph}{\bprot@dottedtocline{7}{9em}{7em}}
\newcommand{\l@fourfoldparagraph}{\bprot@dottedtocline{8}{10em}{7em}}
\newcommand{\l@fivefoldparagraph}{\bprot@dottedtocline{9}{10em}{7em}}
\newcommand{\l@sixfoldparagraph}{\bprot@dottedtocline{10}{10em}{7em}}
\newcommand{\l@sevenfoldparagraph}{\bprot@dottedtocline{11}{10em}{7em}}
\newcommand{\l@eightfoldparagraph}{\bprot@dottedtocline{12}{7em}{3em}}
\newcommand{\l@ninefoldparagraph}{\bprot@dottedtocline{13}{7em}{3em}} 
\makeatother

% für hyperref
\makeatletter 
	\providecommand*{\toclevel@subsubparagraph}{7}
	\providecommand*{\toclevel@fourfoldparagraph}{8}
	\providecommand*{\toclevel@fivefoldparagraph}{9}
	\providecommand*{\toclevel@sixfoldparagraph}{10}
	\providecommand*{\toclevel@sevenfoldparagraph}{11}
	\providecommand*{\toclevel@eightfoldparagraph}{12}
	\providecommand*{\toclevel@ninefoldparagraph}{13}
\makeatother

% Vordefinierte Farben
% ========
% Farben 
% Farbe der Ueberschriften
\definecolor{sectioncolor}{RGB}{0, 0, 0}    % Schwarz
%
% Farben
\definecolor{textcolor}{RGB}{0, 0, 0}        % Schwarz
\definecolor{darkgreen}{RGB}{0,100,0}		 % Dunkelgrün
\definecolor{darkred}{RGB}{139,0,0}			 % Dunkelrot
\definecolor{grey97}{RGB}{247,247,247}		 % grey97
\definecolor{darkgrey}{RGB}{64,64,64}		 % grey97
\definecolor{blau1}{RGB}{135,206,250}		 % Helles Blau
\definecolor{blau2}{rgb}{0.1,0.1,0.25}		 % Dunkleres Blau (alt: 0.5)
\definecolor{orange}{rgb}{1,0.39,0.04}
\definecolor{hellgelb}{rgb}{1.0,1.0,0.9}
\definecolor{hellblau}{rgb}{0.85,0.95,1.0}
\definecolor{dunkelgrau}{gray}{0.35}
\definecolor{mittelgrau}{gray}{0.85}
\definecolor{hellgrau}{gray}{0.93}
\definecolor{link}{rgb}{0.7,0.28,0.0}
%
% Farbe fuer grau hinterlegte Boxen (fuer Paket framed.sty)
\definecolor{shadecolor}{gray}{0.90}

% Farben fuer die Links im PDF
\definecolor{pdfurlcolor}{rgb}{0.6,0,0}
\definecolor{pdffilecolor}{rgb}{0,0,0}
\definecolor{pdflinkcolor}{rgb}{0,0,0}

% sonstige Makros
% griechische Kleinbuchstaben
\newcommand{\alphap}{\mbox{\boldmath$\alpha$}}
\newcommand{\betap}{\mbox{\boldmath$\beta$}}
\newcommand{\gammap}{\mbox{\boldmath$\gamma$}}
\newcommand{\deltap}{\mbox{\boldmath$\delta$}}
\newcommand{\epsilonp}{\mbox{\boldmath$\epsilon$}}
\newcommand{\zetap}{\mbox{\boldmath$\zeta$}}
\newcommand{\etap}{\mbox{\boldmath$\eta$}}
\newcommand{\thetap}{\mbox{\boldmath$\theta$}}
\newcommand{\iotap}{\mbox{\boldmath$\iota$}}
\newcommand{\kappap}{\mbox{\boldmath$\kappa$}}
\newcommand{\lambdap}{\mbox{\boldmath$\lambda$}}
\newcommand{\mup}{\mbox{\boldmath$\mup$}}
\newcommand{\nup}{\mbox{\boldmath$\nup$}}
\newcommand{\xip}{\mbox{\boldmath$\xip$}}
\newcommand{\omicronp}{\mbox{\boldmath$\omicron$}}
\newcommand{\pip}{\mbox{\boldmath$\pi$}}
\newcommand{\rhop}{\mbox{\boldmath$\alpha$}}
\newcommand{\sigmap}{\mbox{\boldmath$\rho$}}
\newcommand{\varsigmap}{\mbox{\boldmath$\varsigma$}}
\newcommand{\taup}{\mbox{\boldmath$\tau$}}
\newcommand{\upsilonp}{\mbox{\boldmath$\upsilon$}}
\newcommand{\phip}{\mbox{\boldmath$\phi$}}
\newcommand{\chip}{\mbox{\boldmath$\chi$}}
\newcommand{\psip}{\mbox{\boldmath$\psi$}}
\newcommand{\omegap}{\mbox{\boldmath$\omega$}}

% für Auflistungen
\newenvironment{Auflistung}
{\begin{itemize}[parsep=0.2em,leftmargin=*]}
{\end{itemize}}

% für Aufzählungen
\newenvironment{Aufzaehlung}
{\begin{enumerate}[parsep=0.2em,leftmargin=*]}
{\end{enumerate}}

% für lange Zitate
\newenvironment{bquote}% 
   {\begin{quotation}\small}% 
   {\end{quotation}}% 
\SetBlockEnvironment{bquote}

% für kurze Zitate
\newenvironment{squote}% 
   {\begin{quote}\small}% 
   {\end{quote}}% 
\SetBlockEnvironment{squote}

% Bibliographie und Fußnoten
% ========
% Bibliographie und Fußnoten

% Konfiguriert das jurabib-Paket. Kommentare aus der Dokumentation
\jurabibsetup{%
	authorformat=italic, 		% Autor kursiv
	titleformat={commasep,all},	% Komma zwischen Autor/Bearbeiter und Titel im Zitat;
	annotatorformat=italic, 	% Bearbeiter kursiv
	annotatorlastsep=divis, 	% Bearbeiter nach Bindestrich
	commabeforerest, 			% Komma nach Verfasser (vor dem Rest)
	crossref={long,dynamic}, 	% Lange Querverweise (auf Festschriften etwa)
	howcited=compare, 			%"zitiert als...", wenn shorttitle anders als title
	pages={always,test}, 		% zitierten Seitenbereich immer ausgeben (always),
	bibformat={tabular,ibidem},	% Litverz. tabellarisch, mit der-/dieselbe
	lookforgender, 				% Auf das gender-Feld achten, um ders./dies. Zitate zu ermglichen
	superscriptedition=switch, 	% Hochgestellte Auflage, wenn ssedition=1 in .bib
	dotafter=bibentry, 			% Punkt nach jedem Eintrag im Lit.verzeichnis
}

\citetitlefortype{article,periodical,incollection} % Diese immer mit Titel zitieren
\formatpages[~]{article}{(}{)} % Zeitschriften als JZ 2001, 1057, (S.) %[, ]
\formatpages[~]{incollection}{(}{)} % Sammbelbandbeiträge als FS xy, 1057, (S.) %[, ]

%Bei Festschriften und Zeitschriftenartikeln: "`in"' vor Titel der Sammlung
\renewcommand{\bibjtsep}{In: } 
\renewcommand{\bibbtsep}{In: } 

%Bei Periodika (AcP et.al.) die Jahreszahl in runde (statt eckige) Klammern setzen.
\renewcommand*{\bibpldelim}{(}
\renewcommand*{\bibprdelim}{)}

%Linke Spalte des Lit.verz. soll ein Drittel der ges. Textbreite einnehmen
\renewcommand*{\bibleftcolumn}{\textwidth/3}

%Nicht Punkt, sondern Komma nach Auflage
\DeclareRobustCommand{\jbaensep}{,}

%Bei Artikeln: Heft-Nummer in Klammern hinter dem Erscheinungsjahr, etwa 2002(7). (aus jurabib-Gruppe)
\DeclareRobustCommand{\artnumberformat}[1]{(#1)}

%Kein Komma hinter Zeitschriftenname (aus: jurabib-Gruppe #661)
\AddTo\bibsgerman{\def\ajtsep{}}

% Keine hochgestellten Ziffern in der Fussnote (KOMA-Script-spezifisch):
\deffootnote{1.5em}{1em}{\makebox[1.5em][l]{\thefootnotemark}}

% Abstand Text <-> Fussnote
\addtolength{\skip\footins}{\baselineskip}

% Verhindert das Fortsetzen von Fussnoten auf der gegenüberligenden Seite
\interfootnotelinepenalty=10000

% durch die jura.tex werden verschiedene spezielle Jura Funktionen bereit gestellt
% ========
% Jura

%**************************************************************
% Entnommen jur-rdnr.sty
% Peter Felix Schuster (http://www.peterfelixschuster.de)
%**************************************************************
\makeatletter
% Zähler für Fälle und Abwandlungen definieren und setzen
\newcommand{\@rdnrname}{Rn.\,} % Makro für die Bezeichnung für Fälle, macht es flexibel zu ändern
\newcounter{randnr} % Zähler randnr (für Randnummer)
\newcounter{zwrandnr}[randnr] % Zähler zwrandnr (für Zwischen-Randnummer), eine Erhöhung von randnr setzt ihn zurück
\setcounter{randnr}{0} % Anfangs auf 0 setzen
\setcounter{zwrandnr}{0} % Anfangs auf 0 setzen
\newcommand{\newrandnummer}{\refstepcounter{randnr}} % Makro, das den Zähler erhöht
\newcommand{\newzwischenrandnummer}{\refstepcounter{zwrandnr}} % Makro, das den Zähler erhöht
\renewcommand{\therandnr}{\arabic{randnr} }% In arabischen Ziffern nummerieren.
\renewcommand{\thezwrandnr}{\therandnr\,\alph{zwrandnr}} % Kleinbuchstaben nach ''normaler'' Randnummer.
\renewcommand{\p@randnr}{\@rdnrname} % Bei Verweisen Rdnr. vor Zahl ausgegeben
\renewcommand{\p@zwrandnr}{\@rdnrname} % Bei Verweisen Rdnr. vor Zahl\,Buchstabe ausgegeben

% marginpar Config
\setlength{\marginparwidth}{15mm}
\marginparsep5mm
\normalmarginpar

% Zähler über Makro erhöhen und am Rand als Randnummer ausgeben
\newcommand{\randnummer}{\newrandnummer\marginpar{\vspace{-4pt}{\strut\\\centering\textbf{\therandnr}}}}

% Makro für Zwischenrandnummern
\newcommand{\zwischenrandnummer}{\newzwischenrandnummer\marginpar{\strut\\\centering\textbf{\thezwrandnr}}}
\makeatother
%**************************************************************

% Formatierung für nicht nummerierte Listen 
\makeatletter
\renewcommand{\labelitemi}{$\m@th\bullet$} 
\renewcommand{\labelitemii}{--}
\renewcommand{\labelitemiii}{$\m@th\circ$}
\renewcommand{\labelitemiv}{$\m@th\rightarrow$}
\makeatother

% Formatierung für nummerierte Listen [I, 2, c) dd)]
\renewcommand{\labelenumi}{\Roman{enumi}.}
\renewcommand{\labelenumii}{\arabic{enumii}.}
\renewcommand{\labelenumiii}{\alph{enumiii})}
\renewcommand{\labelenumiv}{\alph{enumiv}\alph{enumiv})}



%Abkürzungen - vornehmlich für schmale Zwischenräume (\,) nützlich
%benötigt für einige Befehle jurabib. Befehl nachschlagen, mit dem überprüft werden kann, ob geladen!
\providecommand{\Abs}[1]{Abs.\,#1}
\providecommand{\aF}{a.\,F.\xspace} %\aF => a. F.
\providecommand{\andM}{\textbf{a.\,M.}\xspace} %\andM => a. M. in Fettdruck
\providecommand{\euro}[1]{#1\,\EUR} %\euro{x} => x (Eurosymbol)
\providecommand{\folg}{\,f.\xspace} %f. nach kurzem Leerraum
\providecommand{\ffolg}{\,ff.\xspace} %ff. nach kurzem Leerraum
\providecommand{\hL}{h.\,L.\xspace} %\hL => h.L.
\providecommand{\hM}{h.\,M.\xspace} %\hM => h. M.
\providecommand{\idR}{i.\,d.\,R.\xspace} %\idR => i. d. R.
\providecommand{\iHv}{i.\,H.\,v.\xspace} %\iHv => i. H. v.
\providecommand{\iSd}{i.\,S.\,d.\xspace} %\iSd => i. S. d.
\providecommand{\iSv}{i.\,S.\,v.\xspace} %\iSv => i. S. v.
\providecommand{\iVm}{i.\,V.\,m.\xspace} %\iVm => i. V. m.
\providecommand{\ja}{\ding{'63}\xspace}%\ensuremath{\bigoplus}%(+) 
\providecommand{\mwN}{m.\,w.\,N.\xspace} %\mwN => m. w. N.
\providecommand{\LWL}{\bfseries\underline{\emph{LWL}}}%steht für Literaturwunschliste, also für Blindzitate. Alles folgende wird fett!
\providecommand{\nein}{\ding{'67}\xspace}%\ensuremath{\ominus}%(-)
\providecommand{\nF}{n.\,F.\xspace} %\nF => n. F.
\providecommand{\Nr}[1]{Nr.\,#1}
\providecommand{\Nro}[1]{Nr.\,#1}
\providecommand{\pg}[1]{\S\,#1} %\pg{x} => (Paragraf) x
\providecommand{\Pg}[1]{\SSS\,#1} %\Pg{x} => (Paragrafen) x 
\providecommand{\pgAbs}[2]{\S\,#1 Abs.\,#2} %\pgAbs{x}{y} => (Paragraf) x Abs. y
\providecommand{\pgAbsS}[3]{\S\,#1 Abs.\,#2 S.\,#3}
\providecommand{\pgRn}[2]{\S\,#1 Rn.\,#2} %\pg{x}{y} => (Paragraf) x Rn. y
\providecommand{\pgS}[2]{\S\,#1 S.\,#2}
\providecommand{\Rn}[1]{Rn.\,#1}
\providecommand{\Satz}[1]{S.\,#1}
\providecommand{\Seite}[1]{S.\,#1}
\providecommand{\zBsp}{z.\,B.}
\providecommand{\fremdwort}[2]{\foreignlanguage{#1}{\itshape{}#2}}%\fremdwort{#1=Sprache}{#2=Text}
\newcommand*{\qll}[1]{\emph{#1}}%           fuer Quellen
\newcommand{\code}[1]{\texttt{#1}}%         fuer Computeranweisungen, tags o.ae.
\newcommand*{\marke}[1]{{\scshape #1}}%     Markennamen % \texttrademark (TM) oder \textregistered (R)?
\newcommand*{\firma}[1]{{\scshape #1}}%     Unternehmensbezeichnung
\newcommand*{\prdbez}[1]{%                  Produktbezeichnung
  {\scshape #1}%
  \index{#1}%
}
\providecommand{\altzahl}[1]{{\oldstyle #1}}
%\DeclareTextSymbol{\textNr}{TS1}{'233}

\newcommand*{\arr}{\(\rightarrow\)\space}
\newcommand*{\arrr}{\(\longrightarrow\)\space}
\newcommand*{\Arr}{\(\Rightarrow\)\space}
\newcommand*{\Arrr}{\(\Longrightarrow\)\space}
\newcommand*{\lrarr}{\(\leftrightarrow\)\space}

\newcommand*{\larr}{\(\leftarrow\)}
\newcommand*{\Larr}{\(\Leftarrow\)}

\newcommand*{\darr}{\(\downarrow\)}
\newcommand*{\Darr}{\(\Downarrow\)}


